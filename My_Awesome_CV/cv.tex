%!TEX TS-program = xelatex
%!TEX encoding = UTF-8 Unicode
% Awesome CV LaTeX Template for CV/Resume
%
% This template has been downloaded from:
% https://github.com/posquit0/Awesome-CV
%
% Author:
% Claud D. Park <posquit0.bj@gmail.com>
% http://www.posquit0.com
%
% Template license:
% CC BY-SA 4.0 (https://creativecommons.org/licenses/by-sa/4.0/)
%


%-------------------------------------------------------------------------------
% CONFIGURATIONS
%-------------------------------------------------------------------------------
% A4 paper size by default, use 'letterpaper' for US letter
\documentclass[11pt, a4paper]{awesome-cv}
\usepackage{hyperref}
\definecolor{mypink1}{rgb}{0,0.2,0.6}
%\hypersetup{
    %colorlinks=true,
   % linkcolor=blue,
    %filecolor=magenta,      
   % urlcolor={mypink1} 
%}
% \usepackage{fontawesome}
\usepackage{fontawesome5}
\usepackage{tabularx}
% \usepackage{fontspec}
% \newfontface\FA{./fonts/FontAwesome.ttf}[Contextuals={WordInitial,WordFinal}]

% Configure page margins with geometry
\geometry{left=1.4cm, top=.8cm, right=1.4cm, bottom=1.8cm, footskip=.5cm}

% Specify the location of the included fonts
% \fontdir[./fonts/]

% Color for highlights
% Awesome Colors: awesome-emerald, awesome-skyblue, awesome-red, awesome-pink, awesome-orange
%                 awesome-nephritis, awesome-concrete, awesome-darknight
\colorlet{awesome}{awesome-red}
% Uncomment if you would like to specify your own color
% \definecolor{awesome}{HTML}{CA63A8}

% Colors for text
% Uncomment if you would like to specify your own color
% \definecolor{darktext}{HTML}{414141}
% \definecolor{text}{HTML}{333333}
% \definecolor{graytext}{HTML}{5D5D5D}
% \definecolor{lighttext}{HTML}{999999}

% Set false if you don't want to highlight section with awesome color
\setbool{acvSectionColorHighlight}{true}

% If you would like to change the social information separator from a pipe (|) to something else
\renewcommand{\acvHeaderSocialSep}{\quad\textbar\quad}


%-------------------------------------------------------------------------------
%	PERSONAL INFORMATION
%	Comment any of the lines below if they are not required
%-------------------------------------------------------------------------------
% Available options: circle|rectangle,edge/noedge,left/right
% \photo{./examples/profile.png}
\name{Mortaza}{Hassani}
% \position{PhD Applicant} %{Research Assistant{\enskip\cdotp\enskip}Indiana University}
\address{Avenue Paul Heger 22, Apartment 672, Brussels 1000, Belgium}

\email{mortaza.hassani@ulb.be}
% \homepage{linkedin.com/in/MortazaHassani}
\github{MortazaHassani}
\linkedin{MortazaHassani}
% \gitlab{gitlab-id}
% \stackoverflow{SO-id}{SO-name}
% \twitter{@twit}
% \skype{skype-id}
% \reddit{reddit-id}
% \medium{madium-id}
% \googlescholar{googlescholar-id}{name-to-display}
%% \firstname and \lastname will be used
% \googlescholar{googlescholar-id}{}
% \extrainfo{Nationality: Afghan}

%\quote{``Be the change that you want to see in the world."}


%-------------------------------------------------------------------------------
\begin{document}

% Print the header with above personal informations
% Give optional argument to change alignment(C: center, L: left, R: right)
\makecvheader

% Print the footer with 3 arguments(<left>, <center>, <right>)
% Leave any of these blank if they are not needed
\makecvfooter
  {\today}
  {Mortaza Hassani~~~·~~~Curriculum Vitae}
  {\thepage}


%-------------------------------------------------------------------------------
%	CV/RESUME CONTENT
%	Each section is imported separately, open each file in turn to modify content
%-------------------------------------------------------------------------------
% \cvsection{Objective}
\begin{rSection}
    \emph{Aspiring PhD applicant in cybersecurity and computer engineering with a focus on Cybersecurity, Embedded Systems and Artificial Intelligence.
    Aiming to contribute to innovative research in the design and security of embedded systems, AI \& ML developments, System-on-Chip
    security, and IoT through a PhD opportunity.}
\end{rSection}
%-------------------------------------------------------------------------------
%	SECTION TITLE
%-------------------------------------------------------------------------------
\cvsection{Education}


%-------------------------------------------------------------------------------
%	CONTENT
%-------------------------------------------------------------------------------
\begin{cventries}

%---------------------------------------------------------
  \cventry
    {M.S. Cybersecurity, IoT Specialization (M2)} % Degree
    {\href{https://www.ulb.be/en/programme/m-secum}{Université Libre de Bruxelles (ULB)}} % Institution
    {Brussels, Belgium} % Location
    {Sept 2024 - Sept 2025} % Date(s)
    {
      \begin{cvitems} % Description(s) bullet points
        \item[] {Thesis: -- To Be Decided --}
      \end{cvitems}
    }
    
%---------------------------------------------------------

\cventryshort
    {M.S. Cybersecurity, IoT Specialization  (M1) (GPA: 15.81/20 | 1\textsuperscript{st} Position)} % Degree
    {\href{https://www.univ-ubs.fr/}{Université Bretagne Sud (UBS)}} % Institution
    {Lorient, France} % Location
    {Sept 2023 - July 2024} % Date(s)
    
%---------------------------------------------------------
\cventryshort
    {B.E. Computer Systems Engineering (GPA: 3.87/4,00 | Gold Medalist | 1\textsuperscript{st} Position)} % Degree
    {\href{https://jinnah.edu/}{Mohammad Ali Jinnah University (MAJU)}} % Institution
    {Karachi, Pakistan} % Location
    {Sept 2019 - July 2023} % Date(s)
    {
      \begin{cvitems} % Description(s) bullet points
        \item[] {Thesis: Robotic Control System using Computer Vision on Embedded System}
      \end{cvitems}
    }
%--------------------------------------------------------- 
\end{cventries}

% %-------------------------------------------------------------------------------
%	SECTION TITLE
%-------------------------------------------------------------------------------
\cvsection{Appointment}


%-------------------------------------------------------------------------------
%	CONTENT
%-------------------------------------------------------------------------------
\begin{cventries}

%---------------------------------------------------------
  \cventry
    {FPGA Based Systems Lab, Faculty of Electrical \& Computer Engineering, Mohammad Ali Jinnah University} % Job title
    {Teacher Assistant} % Organization
    {Karachi, Pakistan} % Location
    {Sept. 2022 - Jan. 2023} % Date(s)
    {
      \begin{cvitems} % Description(s) of tasks/responsibilities
        \item[] \hspace{-1em}{Facilitated lab sessions for undergraduates focused on FPGA-based system design and implementation.}
        \item \hspace{1em}{Provided hands-on guidance for configuring FPGAs, designing digital circuits, and troubleshooting hardware.}
        \item \hspace{1em}{Reinforced theoretical concepts through practical applications, enhancing students' proficiency in FPGA technology.}
        % \item \hspace{1em}{Redesigned the website.}
      \end{cvitems}
    }
\end{cventries}

\cvsection{Research Interests}

\begin{rSection}
    \textit{ Design \& Security of Embedded Systems, Internet of Things, System-on-Chip Security,  Wireless Communication, Hardware Security, AI/ML Development and Application, Robotics, Computer Vision, FPGA}
\end{rSection}
%-------------------------------------------------------------------------------
%	SECTION TITLE
%-------------------------------------------------------------------------------
\cvsection{Skills}


%-------------------------------------------------------------------------------
%	CONTENT
%-------------------------------------------------------------------------------
\begin{cvskills}
%---------------------------------------------------------
  \cvskill
    {Core Skills} % Category
    {Embedded Systems, Cybersecurity, Artificial Intelligence, Machine/Deep Learning, Network, Linux
   , SoC Design, IoT Development, Web Development} % Skills
%---------------------------------------------------------
  \cvskill
    {Security \& Analysis} % Category
    {Vulnerability Assessment, Reconnaissance, Penetration Testing, Threat
    Modeling, Malware \& Reverse Engineering, Secure Coding, Cryptography} % Skills
%---------------------------------------------------------
  \cvskill
    {Programming}
    {Python, C++, C, Bash, Java, PHP, HTML/CSS, SQL, Verilog, SystemVerilog}
%---------------------------------------------------------
  \cvskill
    {Soft Skills}
    {Problem-Solving, Project Management, Leadership, Team Collaboration,
    Effective Communication, Planning, Delivering Results, Agility, Autonomy}
%---------------------------------------------------------
  \cvskill
    {Tools} % Category
    {Git, VS Code, TensorFlow, PyTorch, BurpSuite, LiteX, Vivado, Matlab,
    PlatformIO, Wireshark, AWS, Azure, NMAP, ZAP, Ghidra, UML, Markdown, LaTex } % Skills
%---------------------------------------------------------
\end{cvskills}

%-------------------------------------------------------------------------------
%	SECTION TITLE
%-------------------------------------------------------------------------------
\cvsection{Work Experience}


%-------------------------------------------------------------------------------
%	CONTENT
%-------------------------------------------------------------------------------
\begin{cventries}

%---------------------------------------------------------

  \cventry
      {\href{https://labsticc.fr/en/directory/tanguy-philippe}{Lab-STICC - ARCAD}} % Organization
      {Cybersecurity Intern - Offensive Tool for Studying SoC Communication Interfaces} % Job title
      {Lorient, France} % Location
      {Apr. 2024 - June 2024} % Date(s)
      {
        \begin{cvitems} % Description(s) of tasks/responsibilities
          \item[] \hspace{-1em}{Contributed in research and studying the development of offensive tools to analyze System-on-Chip (SoC) communication interfaces using \href{https://github.com/enjoy-digital/litex}{LiteX framework}.: \vspace{0.2em}}
          \item \hspace{1em}{Developed and implemented a System-on-Chip (SoC) on an FPGA board using the LiteX framework.}
          \item \hspace{1em}{Generated SoC using C and Python for SoC definition and configuration.}
          \item \hspace{1em}{Successfully integrated UART, SPI and I2C communication cores on SoC for sniffing purposes.}
          \item \hspace{1em}{Researched comprehensive security assessments and formulated recommendations for enhancements.}
        \end{cvitems}
      }

%---------------------------------------------------------
\vspace{0.3em}

  \cventry
    {\href{https://lambdatheta.com/}{Lambda Theta}} % Organization
    {Design Engineer} % Job title
    {Karachi, Pakistan} % Location
    {Apr. 2022 - Aug. 2023} % Date(s)
    {
      \begin{cvitems} % Description(s) of tasks/responsibilities
        \item[] \hspace{-1em}{Constructed test environments/units using UVM and OpenFPGA to create customized FPGA architectures: \vspace{0.2em}}
        \item \hspace{1em}{Utilized Verilog, VHDL, and C++ to develop and test designs, ensuring compatibility across platforms.}
        \item \hspace{1em}{formulated SystemVerilog for testbench creation for the verification of designs.}
        \item \hspace{1em}{Achieved improvements in synthesis and performance through developed prototypes.}
      \end{cvitems}
    }

%---------------------------------------------------------

\vspace{0.3em}

  \cventry
    {\href{https://lambdatheta.com/}{Lambda Theta}} % Organization
    {Computer Vision Intern} % Job title
    {Karachi, Pakistan} % Location
    {July 2021 - Oct. 2021} % Date(s)
    {
      \begin{cvitems} % Description(s) of tasks/responsibilities
        \item[] \hspace{-1em}{Engineered human and object detection algorithms using TensorFlow and OpenCV: \vspace{0.2em}}
        \item \hspace{1em}{Enhanced real-time processing capabilities on RaspberryPi, resulting in detection speed and reliability.}
        \item \hspace{1em}{Integrated an industrial object detection algorithm incorporating volumetric calculations to enhance spatial analysis and efficiency. Integrated the algorithm into a courier company's workflow.}
        \item \hspace{1em}{Systematized extensive testing and validation to ensure robust performance in various environments.}
      \end{cvitems}
    }

%---------------------------------------------------------

\end{cventries}

%-------------------------------------------------------------------------------
%	SECTION TITLE
%-------------------------------------------------------------------------------
\cvsection{Teaching Experience}

\begin{rSection}{}{\bodyfont\bfseries\color{black} \hspace{-1.6em} {Mohammad Ali Jinnah University}} \hfill{\bodyfont\slshape\color{awesome} {Karachi, Pakistan} \vspace{-0.5em}}
\end{rSection}

%-------------------------------------------------------------------------------
%	CONTENT
%-------------------------------------------------------------------------------
\begin{cventries}

%---------------------------------------------------------
  \cventry
    {Teacher Assistant} % Job title
    {} % Organization
    {} % Location
    {Sept 2022 - Jan. 2023} % Date(s)
    {
      \begin{cvitems} % Description(s) of tasks/responsibilities
        \item[] {Facilitated lab sessions for undergraduates focused on FPGA-based system design and implementation. Provided hands-on
        guidance for configuring FPGAs, designing digital circuits, and troubleshooting hardware. Reinforced theoretical concepts
        through practical applications, enhancing students’ proficiency in FPGA technology.}
      \end{cvitems}
    }
%---------------------------------------------------------
  % \cventry
  %   {\hspace{-0.2em}Graduate Assistant\vspace{-0.2em}} % Job title
  %   {} % Organization
  %   {} % Location
  %   {Aug. 2018 - Dec. 2019\vspace{0.4em}} % Date(s)
  %   {
  %     \begin{cvitems} % Description(s) of tasks/responsibilities
  %       \item[] \hspace{-0.2em}{Assistant for Matt Pierce in \textit{MSCH C207 Introduction to Media Industry and Management}, 125 students \vspace{0.3em}}
  %       \item[] \hspace{-0.2em}{Assistant for Nathan Mishler in \textit{MSCH C200 Videogame Industry: System and Management}, 60 students \vspace{0.7em}}
  %     \end{cvitems}
  %   }   


\begin{rSection}{}{\bodyfont\bfseries\color{black} \hspace{-1.6em} {Internationa Assistance Mission}} \hfill{\bodyfont\slshape\color{awesome} {Herat, Afghanistan} \vspace{0.3em}}
\end{rSection}

%---------------------------------------------------------
  \cventry
    {Computer Instructor \vspace{0.1em}} % Job title
    {} % Organization
    {} % Location
    {\vspace{-0.1em}Apr. 2017 - June 2017} % Date(s)
    {
      \begin{cvitems} % Description(s) of tasks/responsibilities
        \item[] {Taught database management, data entry, record keeping, and report generation to IAM's staff and employees.}
        \item[] {Developed and delivered comprehensive training sessions on querying databases and generating reports.}
        \item[] {Provided hands-on training and support to ensure staff proficiency in using database management systems.}
      \end{cvitems}
    }
%---------------------------------------------------------
\end{cventries}


%-------------------------------------------------------------------------------
%	SECTION TITLE
%-------------------------------------------------------------------------------
\cvsection{Honors \& Awards}


\begin{rSection}{}
  {\hspace{-0.5em}2024\hspace{1em}}{\bodyfont\bfseries\color{darktext} {Gold Medalist}}{ , Academic Award , Mohammad Ali Jinnah University
  } \hfill{\bodyfont\slshape\color{awesome} {Karachi, Pakistan}}
  \\Highest Academic Achievement and 1\textsuperscript{st} Position In Computer Engineering Department Batch 2019.
  %1 \textsuperscript{st}
  \end{rSection}

  
\begin{rSection}{}
{\hspace{-0.5em}2023\hspace{1em}}{\bodyfont\bfseries\color{darktext} {Erasmus Mundus Joint Master Degree Scholarship}}{ , Merit Award ``\href{https://master-cyberus.eu/programme/overview}{CYBERUS}
} \hfill{\bodyfont\slshape\color{awesome} {France, Estonia, Belgium}}
\\Shortlisted \& Selected Among 480+ International Applicants .
%1 \textsuperscript{st}
\end{rSection}

\begin{cvhonors} 


%---------------------------------------------------------
  \cvhonor
    {Embedded CTF Runner-Up} % Award
    {CYBERUS Spring School} % Event
    {\hspace{-2em}\entrylocationstyle {Lorient, France}} % Location
    {2024} % Date(s)
    
    
%---------------------------------------------------------
  \cvhonor
    {Best Engineering Project} % Award
    {FYP Project Exhibition, Mohammad Ali Jinnah University} % Event
    {\hspace{-2em}\entrylocationstyle{Karachi, Pakistan}} % Location
    {2023} % Date(s)

%---------------------------------------------------------
  \cvhonor
    {Chancellor Honor Roll List} % Award
    {Academic Achievement Ceremony} % Event
    {\hspace{-2em}\entrylocationstyle{Karachi, Pakistan}} % Location
    {2022} % Date(s)

%---------------------------------------------------------
  \cvhonor
    {1\textsuperscript{st} Position Samsung Innovation Campus - AI} % Award
    {SAMSUNG \& MAJU} % Event
    {\hspace{-2em} \entrylocationstyle {Karachi, Pakistan}} % Location
    {2021} % Date(s)

%---------------------------------------------------------
  \cvhonor
    {Chancellor Honor Roll List} % Award
    {Academic Achievement Ceremony} % Event
    {\hspace{-2em}\entrylocationstyle{Karachi, Pakistan}} % Location
    {2021} % Date(s)

%---------------------------------------------------------
  \cvhonor
    {Chancellor Honor Roll List} % Award
    {Academic Achievement Ceremony} % Event
    {\hspace{-2em}\entrylocationstyle{Karachi, Pakistan}} % Location
    {2020} % Date(s)

% %---------------------------------------------------------

\end{cvhonors}




%-------------------------------------------------------------------------------
%	SECTION TITLE
%-------------------------------------------------------------------------------
\cvsection{Projects}

%-------------------------------------------------------------------------------
%	CONTENT
%-------------------------------------------------------------------------------

\begin{cventries}
%---------------------------------------------------------
\begin{rSection}{}
{\vspace{-0.5em}\hspace{-1.5em}[P1]\hspace{1em}\bodyfont\bfseries\color{darktext} {RoboPot: Autonomous Golf Ball Potting Robot \hspace{1em} \href{https://github.com/MortazaHassani/RoboPutter}{\Large{\faGithub}}}}
\\\hspace{1.2em}Designed an autonomous robot to enhance precision in golf ball potting. Integrated a PID controller for movement, computer vision for target detection, and a Raspberry Pi for processing. Used MQTT for communication and NodeMCU for actuation.
\\ \faIcon{trophy} \hspace{0.5em} \emph{1\textsuperscript{st} place in the Final Year Project (FYP) exhibition}
% \\\hspace{1.6em} \textbf{Highlights}:\hspace{1em} \href{https://c101.hongtaoh.com/feedback/}{Students feedback,} \hspace{1em} \href{https://c101.hongtaoh.com/slides/}{Slides,} \hspace{1em} \href{https://c101.hongtaoh.com/videos/}{Lecture videos}
\vspace{-0.2em}
\end{rSection}

%---------------------------------------------------------
\begin{rSection}{}
{\hspace{-1.5em}[P2]\hspace{1em}}{\bodyfont\bfseries\color{darktext} {Real-Time Exchange Rate Display \hspace{1em} \href{https://www.linkedin.com/posts/mortazahassani_currency-rates-board-uml-and-images-activity-7166390758810312704-n4Oi} {\Large{\faLaptop}}}}
\\\hspace{1.2em}Engineered an embedded system with an HDMI interface for displaying real-time exchange rates. Developed using Qt for the GUI, incorporating web scraping for data retrieval. Connected to AWS and Firebase for data management and cloud operations. Real-time updates were sent to Telegram and WhatsApp channels for continuous access.
\vspace{-0.2em}
\end{rSection}

%---------------------------------------------------------
\begin{rSection}{}
{\hspace{-1.5em}[P3]\hspace{1em}}{\bodyfont\bfseries\color{darktext} {IoT-Enabled Campus Scheduling System \hspace{1em} \href{https://maju-iot.web.app/}{\large{\faGlobe}}}}
\\\hspace{1.2em}Developed a web-scheduled campus bell alarm system for Mohammad Ali Jinnah University using NodeMCU modules and a Raspberry Pi. Designed a web interface for schedule management, with data and control functionalities hosted on Google Firebase. This IoT solution enabled automated and precise schedule control.
\end{rSection}

\end{cventries}
\cvsection{Certifications}

    \begin{cvhonors}
        %---------------------------------------------------------
        \cvhonor
            {\href{https://www.credly.com/badges/fef42096-4e09-41ea-9caf-c1b908d23658/public_url}{LFEL1010: XSS Exploits and Defenses}} % Award
            {\href{https://training.linuxfoundation.org/}{The Linux Foundation}} % Event
            {\hspace{-2em}\entrylocationstyle {}} % Location
            {2024} % Date(s)    
        %---------------------------------------------------------
        \cvhonor
            {\href{https://badgr.com/public/assertions/196LrlAyQcmY8NvMpsGKNQ}{Postman API Fundamentals Student Expert}} % Award
            {\href{https://postman.com/}{Postman}} % Event
            {\hspace{-2em}\entrylocationstyle {}} % Location
            {2024} % Date(s)  
         %---------------------------------------------------------
         \cvhonor
            {\href{https://www.coursera.org/account/accomplishments/specialization/certificate/4P9VR6DR3KXA}{Google IT Automation with Python specialization}} % Award
            {\href{https://www.coursera.org/}{Coursera \& Google}} % Event
            {\hspace{-2em}\entrylocationstyle {}} % Location
            {2023} % Date(s)   
        
        \cvhonor
            {7th International Entrepreneurship Summer School}
            {\href{https://ced.iba.edu.pk/}{IBA – Center for Entrepreneurial Development }}
            {\hspace{-2em}\entrylocationstyle {IBA - Karachi, Pakistan}} 
            {2022}
        \cvhonor
            {Artificial Intelligence}
            {\href{https://www.samsung.com/pk/innovation-campus/}{SAMSUNG Innovation Campus}}
            {\hspace{-2em}\entrylocationstyle {Karachi, Pakistan}} 
            {2021}
        \cvhonor
            {Machine Learning with Python}
            {Mohammad Ali Jinnah University}
            {\hspace{-2em}\entrylocationstyle {Karachi, Pakistan}} 
            {2020}
    \end{cvhonors}
    
% \input{cv/events.tex}
%-------------------------------------------------------------------------------
%	SECTION TITLE
%-------------------------------------------------------------------------------
\cvsection{Extra-Curricular}

%-------------------------------------------------------------------------------
%	CONTENT
%-------------------------------------------------------------------------------
\begin{cventries}

%---------------------------------------------------------
  \cventry
    {Google Developers Student Club - Founder \& Lead} % Job title
    {Mohammad Ali Jinnah University - GDSC Chapter} % Organization
    {Karachi, Pakistan} % Location
    {2022 - 2023} % Date(s)
    {
      % \begin{cvitems} % Description(s) of tasks/responsibilities
      %   \item[] {Mohammad Ali Jinnah University - GDSC Chapter}
      % \end{cvitems}
    }
%---------------------------------------------------------
  % \cventry
  %   {\hspace{-0.2em}Graduate Assistant\vspace{-0.2em}} % Job title
  %   {} % Organization
  %   {} % Location
  %   {Aug. 2018 - Dec. 2019\vspace{0.4em}} % Date(s)
  %   {
  %     \begin{cvitems} % Description(s) of tasks/responsibilities
  %       \item[] \hspace{-0.2em}{Assistant for Matt Pierce in \textit{MSCH C207 Introduction to Media Industry and Management}, 125 students \vspace{0.3em}}
  %       \item[] \hspace{-0.2em}{Assistant for Nathan Mishler in \textit{MSCH C200 Videogame Industry: System and Management}, 60 students \vspace{0.7em}}
  %     \end{cvitems}
  %   }   

%---------------------------------------------------------
  \cventry
    {Active Member \& Organizer} % Job title
    {IEEE \& IEEE Young Professionals - Mohammad Ali Jinnah University IEEE Chapter} % Organization
    {Karachi, Pakistan} % Location
    {\vspace{-0.5em} 2021 - 2023} % Date(s)
    {
      % \begin{cvitems} % Description(s) of tasks/responsibilities
      %   \item[] {Mohammad Ali Jinnah University - IEEE Chapter}
      % \end{cvitems}
    }
%---------------------------------------------------------

%---------------------------------------------------------
\cventry
{Volunteer \& Organizer} % Job title
{Robotic \& Engineering Society - Mohammad Ali Jinnah University FOE Dept.} % Organization
{Karachi, Pakistan} % Location
{\vspace{-0.5em} 2020 - 2023} % Date(s)
{
  % \begin{cvitems} % Description(s) of tasks/responsibilities
  %   \item[] {Mohammad Ali Jinnah University - IEEE Chapter}
  % \end{cvitems}
}
%---------------------------------------------------------
\vspace{-2em}
\end{cventries}


%-------------------------------------------------------------------------------
%	SECTION TITLE
%-------------------------------------------------------------------------------
\cvsection{Volunteer-Experience}

%-------------------------------------------------------------------------------
%	CONTENT
%-------------------------------------------------------------------------------
\begin{cventries}

%---------------------------------------------------------
  \cventry
    {Community Safety Planning (CSP) Committee - Director} % Job title
    {Danish Demining Group (DDG) } % Organization
    {Herat, Afghanistan} % Location
    {Oct. 2018 - May 2019} % Date(s)
    {
      % \begin{cvitems} % Description(s) of tasks/responsibilities
      %   \item[] {Mohammad Ali Jinnah University - GDSC Chapter}
      % \end{cvitems}
    }
%---------------------------------------------------------

%---------------------------------------------------------
\vspace{-2em}
\end{cventries}


\cvsection{Languages}

\begin{center}
    \hspace{2em}
    \begin{tabularx}{\textwidth}{@{}X X X@{}}
        \begin{minipage}[t]{\linewidth}
            \languagebar{Persian}{Native}{3}{Proficient}
        \end{minipage} &
        \begin{minipage}[t]{\linewidth}
            \languagebar{English}{C1}{2.5}{Proficient (IELTS 7.0)}
        \end{minipage} &
        \begin{minipage}[t]{\linewidth}
            \languagebar{French}{A2}{1}{Elementary}
        \end{minipage} \\
    \end{tabularx}
\end{center}

% \input{cv/writing.tex}

% \cvsection{Selective Peer-Reviewed Conference Presentations}

\begin{rSection}{}
%--copy and paste this region  if you need more--
\textbf{Hao, H}. (Nov., 2019). \textit{Are Chinese selfies gender-stereotypical: A content analysis of selfies on Weibo}. Paper presented at the 105th Annual Conference of the National Communication Association, Baltimore, MD. \\[3pt]
\textbf{Hao, H}. (Nov., 2019). \textit{Digital divide: Theoretical review and future research suggestions}. Paper presented at the 105th Annual Conference of the National Communication Association, Baltimore, MD. \\[3pt]
\textbf{Hao, H}. (May, 2019).\textit{Global expansion of China’s media for soft power promotion}. Paper presented at the 69th Annual Conference of the International Communication Association, Washington, D.C. 
\begin{itemize}
  \item[] \vspace{-0.5em} \faIcon{trophy} \hspace{0.5em} \emph{Second Top Student Paper}
\end{itemize}

%--copy and paste this region  if you need more--
\end{rSection}
% \input{cv/committees.tex}
\cvsection{References}
\begin{center}
    \hspace{2em}
    \begin{tabularx}{\textwidth}{@{}X X@{}}
        \begin{minipage}[t]{\linewidth}
            \reference{Prof. Guy Gogniat}{Professor, Université Bretagne Sud}{guy.gogniat@univ-ubs.fr}
        \end{minipage} &
        \begin{minipage}[t]{\linewidth}
            \reference{Prof. Philippe Tanguy}{Professor, Université Bretagne Sud}{philippe.tanguy@univ-ubs.fr}
        \end{minipage} \\
        \begin{minipage}[t]{\linewidth}
            \reference{Dr. Syed Muhammad Ghazanfar Monir}{Professor, Karachi School of Business and Leadership}{monir@lambdatheta.com}
        \end{minipage} &
    \end{tabularx}
\end{center}




%-------------------------------------------------------------------------------
\end{document}

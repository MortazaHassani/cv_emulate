%-------------------------------------------------------------------------------
%	SECTION TITLE
%-------------------------------------------------------------------------------
\cvsection{Projects}

%-------------------------------------------------------------------------------
%	CONTENT
%-------------------------------------------------------------------------------

\begin{cventries}
%---------------------------------------------------------
\begin{rSection}{}
{\vspace{-0.5em}\hspace{-1.5em}[P1]\hspace{1em}\bodyfont\bfseries\color{darktext} {RoboPot: Autonomous Golf Ball Potting Robot \hspace{1em} \href{https://github.com/MortazaHassani/RoboPutter}{\Large{\faGithub}}}}
\\\hspace{1.2em}Designed and developed an autonomous robot aimed at enhancing
precision in golf ball potting. The system integrated a PID controller for precise movement control, computer vision
algorithms for target detection, and a Raspberry Pi as the central processing unit. Communication between components
was facilitated through the MQTT protocol, with a NodeMCU managing the actuation mechanisms. 
\\ \faIcon{trophy} \hspace{0.5em} \emph{1\textsuperscript{st} place in the Final Year Project (FYP) exhibition}
% \\\hspace{1.6em} \textbf{Highlights}:\hspace{1em} \href{https://c101.hongtaoh.com/feedback/}{Students feedback,} \hspace{1em} \href{https://c101.hongtaoh.com/slides/}{Slides,} \hspace{1em} \href{https://c101.hongtaoh.com/videos/}{Lecture videos}
\vspace{-0.2em}
\end{rSection}

%---------------------------------------------------------
\begin{rSection}{}
    {\hspace{-1.5em}[P2]\hspace{1em}}{\bodyfont\bfseries\color{darktext} {Robotic Arm Controller Using FPGA \hspace{1em} \href{} {\Large{}}}}
    \\\hspace{1.2em}Engineered and developed mechanism to control 4 axis robot arm using FPGA. The project consisted of PWM generation, clock divider and other algorithms on FPGA board using Verilog.
    \vspace{-0.2em}
    \end{rSection}

%---------------------------------------------------------
\begin{rSection}{}
{\hspace{-1.5em}[P3]\hspace{1em}}{\bodyfont\bfseries\color{darktext} {Real-Time Exchange Rate Display \hspace{1em} \href{https://www.linkedin.com/posts/mortazahassani_currency-rates-board-uml-and-images-activity-7166390758810312704-n4Oi} {\Large{\faLaptop}}}}
\\\hspace{1.2em}Engineered an embedded system with an HDMI interface for displaying real-time exchange rates. Developed using Qt for the GUI, incorporating web scraping for data retrieval. Connected to AWS and Firebase for data management and cloud operations. Real-time updates were sent to Telegram and WhatsApp channels for continuous access.
\vspace{-0.2em}
\end{rSection}

%---------------------------------------------------------
\begin{rSection}{}
    {\hspace{-1.5em}[P4]\hspace{1em}}{\bodyfont\bfseries\color{darktext} {Cryptocurrency Price Forecasting using Machine Learning (LSTM) \hspace{1em} \href{https://www.datacamp.com/datalab/w/59d76458-67c7-4a1f-bbcf-b8adb52c171e#cryptocurrency-price-forecasting-using-machine-learning-lstm} {\Large{\faLaptop}}}}
    \\\hspace{1.2em}Developed a deep learning model using Long Short-Term Memory (LSTM), a specialized type of recurrent neural network (RNN), to forecast cryptocurrency prices. Leveraged LSTM's ability to capture long-term dependencies in sequential data, making it well-suited for time-series analysis. The project aimed to improve the accuracy of cryptocurrency price predictions, offering valuable insights for financial decision-making and market analysis.
    \\ \faIcon{trophy} \hspace{0.5em} \emph{1\textsuperscript{st} place in SAMSUNG Innovation Campus - AI}
    \vspace{-0.2em}
    \end{rSection}
%---------------------------------------------------------
\begin{rSection}{}
{\hspace{-1.5em}[P5]\hspace{1em}}{\bodyfont\bfseries\color{darktext} {IoT-Enabled Campus Scheduling System \hspace{1em} \href{https://maju-iot.web.app/}{\large{\faGlobe}}}}
\\\hspace{1.2em}Developed a web-scheduled campus bell alarm system for Mohammad Ali Jinnah University using NodeMCU modules and a Raspberry Pi. Designed a web interface for schedule management, with data and control functionalities hosted on Google Firebase. This IoT solution enabled automated and precise schedule control.
\end{rSection}

\end{cventries}